\documentclass{beamer}
\usepackage[utf8]{inputenc}
\usepackage[T1]{fontenc}
\title{\LaTeX -Beamer Theme TH Köln}
\institute[TH Köln]{Technische Hochschule Köln}
\date{\today}
\author[de Koster]{Markus de Koster}

\usetheme{thk}
\usepackage[ngerman]{datetime}
\begin{document}

\begin{frame}
\titlepage
\end{frame}

\begin{frame}
    \frametitle{Inhalt}
    \tableofcontents
\end{frame}

\section{Wie sehen Aufzählungen aus?}\label{sec:enumerations}
\begin{frame} 
\frametitle{Wie sehen Aufzählungen aus?} 
\framesubtitle{Ein Beispiel für Aufzählungen} 
\begin{theorem}
    Hier steht ein theorem
\end{theorem} 
\begin{enumerate} 
    \item Hier steht ein item 
    \begin{enumerate} 
        \item Hier steht ein subitem
        \begin{enumerate} 
            \item Hier steht ein subsubitem
        \end{enumerate}
    \end{enumerate}
    \item Hier steht noch ein etwas längeres item, um den Zeilenumbruch zu zeigen. 
    Dieser wird automatisch gesetzt, sobald die Zeile zu lang wird.
\end{enumerate}
\end{frame}

\subsection{Wie sehen Bulletpoints aus?}\label{sec:itemize}
\begin{frame}{Wie sehen Aufzählungen aus?}
\begin{itemize}
\item one
\begin{itemize}
    \item two
    \begin{itemize}
        \item three
    \end{itemize}    
\end{itemize}    
\item four
\end{itemize}
\end{frame}

\section{Eine weitere Section}\label{sec:another_section}
\begin{frame}{Eine weitere Section}
    \begin{itemize}
    \item one
    \begin{itemize}
        \item two
        \begin{itemize}
            \item three
        \end{itemize}    
    \end{itemize}    
    \item four
    \end{itemize}
\end{frame}

\end{document}