\documentclass{beamer}
\usepackage[utf8]{inputenc}
\usepackage[T1]{fontenc}
\usepackage[ngerman]{datetime}
\renewcommand{\dateseparator}{.} % use dots instead of slashes to seperate dates
\usetheme{thk}
% custom colors:
% thk-red
% thk-orange
% thk-violet


\title{\LaTeX -Beamer Theme TH Köln}
\subtitle[Beispielpräsi]{Eine Beispielpräsentation}
\institute[TH Köln]{Technische Hochschule Köln}
% if a short version is given, this will be preferred in the footer. To use the long version, don't give a short version
\supervisor{Prof. Stefan Herzig} %leave empty if not applicable
% you can also use this tag for different purposes, e.g. to provide your email address
\date[\ddmmyyyydate\today]{\today} %short date in brackets, long date in bracelets
\author[de Koster]{Markus de Koster}
\pagefooterlocalization{Seite } % can be anything, e.g. "Slide" "Page" "Seite"

\begin{document}

\begin{frame}
    \vspace{2em} %some distance to the top
    \includegraphics[width=\textwidth]{figures/thk.jpg}
    \titlepage
\end{frame}

\begin{frame}
    \frametitle{Inhalt}
    \tableofcontents
\end{frame}

\section{Aufzählungen}\label{sec:enumerations}
\begin{frame} 
    \frametitle{Aufzählungen} 
    \framesubtitle{Ein Beispiel für Aufzählungen} 
    \begin{enumerate} 
        \item Hier steht ein item 
        \begin{enumerate} 
            \item Hier steht ein subitem
            \begin{enumerate} 
                \item Hier steht ein subsubitem
            \end{enumerate}
        \end{enumerate}
        \item Hier steht noch ein etwas längeres item, um den Zeilenumbruch zu zeigen. 
        Dieser wird automatisch gesetzt, sobald die Zeile zu lang wird.
    \end{enumerate}
\end{frame}

\subsection{Bulletpoints}\label{sec:itemize}
\begin{frame}[allowframebreaks]{Bulletpoints}
    \begin{itemize}
        \item Die erste Stufe ist in rot
        \begin{itemize}
            \item die zweite in orange
            \begin{itemize}
                \item die dritte in Violet
                \item mehr als drei mal darf nicht indentiert werden
            \end{itemize}    
            \item Die Schrift wird je Stufe etwas kleiner
        \end{itemize}    
        \item Weitere bulletpoints haben die selbe Farbe
        \item Dadurch, dass am Beginn dieses Frames das Attribut "allowframebreaks" gesetzt wurde, wird automatisch eine neue Slide geöffnet, wenn diese voll ist
        \begin{itemize}
            \item dafür müssen wir allerdings den "itemize" Bereich verlassen
            \item \dots
        \end{itemize}
    \end{itemize}
    Hier steht Text außerhalb von Bulletpoints. Wenn dieser Text lang genug ist, wird er auf die nächste Seite geschoben.
    Dadurch wird automatisch im Titel der Seite ein Zähler eingefügt. 
    Innerhalb eines itemize erfolgt kein Seitenumbruch. Stattdessen werden die items skaliert. Das selbe gilt für Grafiken, Tabellen, etc.
\end{frame}

\section{Tabellen}\label{sec:tables}
\begin{frame} 
    \frametitle{Tabellen} 
    \begin{table}[ht]
        \centering
        \begin{tabular}{l | l }
          \textbf{Spalte 1}   & \textbf{Spalte 2} \\
          \hline
          SHA-1 &  20 bytes \\ 
          \hline
          SHA-256 &  32 bytes \\ 
          \hline
          SHA-512 &  64 bytes \\ 
          \hline
          MD5 &  16 bytes \\ 
        \end{tabular}
        \caption{\label{tab:example_table} Eine simple Beispieltabelle}
    \end{table}

\end{frame}

\section{Grafiken}\label{sec:graphics}
\begin{frame}{Grafiken}
    \begin{figure}\label{thk-linien}
        \includegraphics[width=\textwidth]{figures/thk-lines.png}
        \caption[TH Köln lines]{TH Köln lines. Adaptiert von \cite{source}}
    \end{figure}
\end{frame}

\section{Blöcke und Theoreme}\label{sec:block}
\begin{frame} 
    \frametitle{Blöcke und Theoreme} 
    \begin{theorem}
        Hier steht ein theorem in default Farbe
    \end{theorem}
	
	\begin{variableblock}{Variabler Block}{bg=white, fg=black}{bg=thk-red, fg=white}
		Hier steht ein variabler Block. Die Farbe wird bei der Deklaration gesetzt.
	\end{variableblock}

    \begin{variableblock}{Variabler Block andere Farbe}{bg=white, fg=black}{bg=thk-orange, fg=white}
        Hier steht ein zweiter variabler Block in anderer Farbe. Es ist länger als eine Zeile, daher erfolgt ein automatischer Zeilenumbruch
    \end{variableblock}
\end{frame}

\end{document}